\documentclass[aps,twocolumn]{revtex4-1}

% Maths
\usepackage{bm}
\usepackage{amsmath}

% Text
\usepackage{siunitx}

% Figures
\usepackage{graphicx}
\usepackage{subfig}

% Referencing
\usepackage{cleveref}
\newcommand{\ccite}[1]{ref.~\cite{#1}}
\newcommand{\Ccite}[1]{Ref.~\cite{#1}}
\newcommand{\ccites}[1]{refs.~\cite{#1}}
\newcommand{\Ccites}[1]{Refs.~\cite{#1}}

% Temp
\usepackage{lipsum}
\usepackage{xcolor}
\newcommand{\temp}[1]{{\leavevmode\color{red}#1}}
\newcommand*{\newlinecommand}[2]{%
  \newcommand*{#1}{%
    \begingroup%
    \escapechar=`\\%
    \catcode\endlinechar=\active%
    \csname\string#1\endcsname%
  }%
  \begingroup%
  \escapechar=`\\%
  \lccode`\~=\endlinechar%
  \lowercase{%
    \expandafter\endgroup
    \expandafter\def\csname\string#1\endcsname##1~%
  }{\endgroup#2\space}%
}
\newlinecommand{\topic}{#1}%\textbf{#1}}



\begin{document}
\title{Efficient saddle point search - Binary image transistion state search}
\author{Sam Avis}
\author{Jack Panter}
\author{Halim Kusumaatmaja}

\begin{abstract}
\lipsum[1]
\end{abstract}

\maketitle


%\section{Introduction \& Method}
\topic Transition states provide fundamental insight into the stability and possible reconfiguration of systems in chemistry, condensed matter physics, and engineering.
Historically, the development of computational methods for locating transition states in free or potential energy landscapes has focused significantly upon particulate and atomistic systems.
In these, the energy barrier between a local energy minimum and a transition state informs the probability of the transition occurring within a thermalised population, and ultimately reaction rates and kinetics \temp{[cite]}.
The energy barriers can also be used to characterise the behaviour of molecules and structures with large numbers of stable configurations, which has applications to protein folding \cite{Bryngelson1995,Onuchic1997}, crystalisation and clusters \cite{Wales1998,Wales2012}.

In addition to these microscale applications it is also increasingly being recognised that transition states are useful in mesoscale or macroscale systems.
Here, the minimum energy barriers provide important lower bounds to the energy input required for transitions to occur.
This is essential to capture in, for example, structural engineering applications, where the minimum energy mechanism for structural failure must be known in order to ensure safety \cite{Panter2019,Hutchinson2018}.
Moreover, it is becoming desirable to functionalise elastic deformation transition mechanisms to enable technologies such as advanced deployable structures \cite{Filipov2015,Zhai2018}, mechanical sensors and actuators, and energy absorption \cite{Shan2015} to name but a few.
Probing minimum energy transition mechanisms are also essential to the development of super liquid-repellent surfaces \cite{Zhang2014,Panter2019b}, and are even finding applications in robotics in efficient locomotion through complex terrain \cite{Othayoth2020}.

\topic Existing transition state search methods generally fall into two categories, single- and double-ended methods.
Single-ended methods are initialised at a single state, typically a minimum in the potential energy landscape, and attempt to identify nearby saddle points.
These methods can converge to a transition state quickly, although are not well suited to identifying specific pathways because a single minimum can be surrounded by many saddle points.
Examples include eigenvector following \cite{Cerjan1981}, the dimer method \cite{Heyden2005,Kastner2008}, and climbing image methods \cite{E2007,Ren2013}.
Double-ended methods, such as the string method \cite{E2002,E2007} and doubly-nudged elastic band (DNEB) \cite{Trygubenko2004}, involve initialising a chain of states between two minima and then minimising the total energy of the chain.
In addition to approximating the transition state, these are useful for obtaining an estimate for the full transition pathway.

\topic Despite this wide variety of existing methods, there are still some important classes of problems relevant to both micro- and macro-scale applications that have not been adequately addressed.
One key problem involves systems that have extremely large numbers of degrees of freedom, such as large particulate systems or discretisations of 3D continuum systems.
In this case, double-ended methods can become computationally expensive, or even prohibitive, owing to the tens of states that are required to accurately approximate the transition state.
A method to reduce the computational overhead in many macro-scale simulations is adaptive coarse-graining or remeshing. \temp{[cite]}
However, for a transition state search method to utilise this it must be able to deal with changing numbers of degrees of freedom as the method runs.
This is a particular problem for double-ended methods for which each of the states will have a different discretisation.
Further issues arise for systems with short-range contact forces, such as hard-core particles \temp{[cite]} and patchy colloids \temp{[cite]}.
In this case, in many configurations there will exist soft-modes where the landscape is locally flat. \temp{[cite]}
This can cause issues for single-ended search methods that rely only upon local information.
Finally, if short range interactions are chosen that are discontinuous the gradients will be undefined, in which case typical landscape methods that require continuous, differentiable optimisation functions will not be appropriate.

\topic In this work we introduce a new method, the Binary Image Transition State Search (BITSS), and demonstrate how it addresses each of these challenges.
This method combines aspects of the two traditional categories; it attempts to find a transition state that separates two local minima using two states that converge from either side.
The states, $\bm{x}_1$ and $\bm{x}_2$, are first initialised in different basins of attraction of the local minima.
While these can be set to the minima, this is not a necessary requirement.
Upon application of the BITSS method, the two images are brought closer to the ridge separating the basins of attraction, while also minimising the energy of the pair of images.
In this way, the two images meet at the lowest energy point on the ridge - the transition state.
This is demonstrated for a simple 2D potential with two minima in \cref{fig:toy2d}a.
The two states begin at the blue minima and follow the red paths as their separation is reduced to converge at the orange transition state.

\begin{figure}[htb]
  \centering
  \includegraphics{fig/toy2d.pdf}
  \caption{
    (a) A simple 2D potential with 2 minima (blue) and an intermediate transition state (orange).
        The red line shows the trajectories of the two states using the BITSS method and the dashed line is the minimum energy pathway.
    (b) A snapshot of the BITSS minimisation showing the driving forces on each state due to the energy constraint, $\bm{F}_E$, and distance constraint, $\bm{F}_D$, with $E_1 < E_2$ and $d(\bm{x}_1,\bm{x}_2) > d_i$.
    (c) The final configuration of the BITSS method showing the two states in red, the transition state in orange, and the negative curvature eigenvector, $\bm{\hat{\tau}}$.
  }
  \label{fig:toy2d}
\end{figure}

\topic Two soft constraints are used to achieve the transition state convergence, while preventing either from passing over the dividing ridge.
The former is implemented by constraining the distance between the states and iteratively reducing the target separation, $d_i$, from the separation of the initial states, $d_0$, by a constant factor:
\begin{equation}
  d_{i+1} = (1 - f) d_i.
  \label{eq:diteration}
\end{equation}
We find that a factor of $f = 0.5$ is succesful for most applications, but this can be made smaller to ensure that the states do not slide off the ridge.
The second constraint sets the energies of the two states to be equal.
This ensures neither state can pass over the ridge so long as the distance is not reduced too quickly.
These constraints are implemented using quadratic energy penalty terms, resulting in the following potential for the pair of states,
\begin{multline}
  V_\text{BITSS}(\bm{x}_1, \bm{x}_2) = E_1 + E_2
    + \kappa_e \left( E_1 - E_2 \right) ^2 \\
    + \kappa_d \left( \mathrm{d}(\bm{x}_1, \bm{x}_2) - d_i \right) ^2,
  \label{eq:bitss}
\end{multline}
where $E_1$ and $E_2$ are the single state energies, and $\mathrm{d}(\bm{x}_1, \bm{x}_2)$ is the measure of distance between the two states.
\Cref{fig:toy2d}b demonstrates the driving forces resulting from each of these energy penalty terms for a 2D potential.

\topic The method is sensitive to the choice of the constraint coefficients, $\kappa_d$ and $\kappa_e$, with optimal values depending upon the specific problem.
Also, as the separation between the states decreases the coefficients will have to change, making them difficult to assign manually.
Consequently, we automate their assignment using information from the system.
They are initially calculated at the start of each minimisation, and regularly recalculated throughout (once per 100 iterations are used in this work).
The coefficients are chosen such that each term in the gradient of \cref{eq:bitss} is of a similar size, providing the following expressions,
\begin{gather}
  \kappa_e = \frac {\alpha} {2 \Delta E},
  \label{eq:ke}
  \\
  \kappa_d = \text{max} \left(
    \frac {\sqrt{|\bm{\nabla} E_1|^2 + |\bm{\nabla} E_2|^2}} {2\sqrt{2} \beta d_i} \; , \;
    \frac{\Delta E}{\beta d_i^2} \right),
  \label{eq:kd}
\end{gather}
where $\alpha$ and $\beta$ are parameters with recommended values of $\alpha = 10$ and $\beta = 0.1$ (see supplementary information).
$\Delta E$ is an estimation for the current energy barrier which is evaluated by linearly interpolating between the two current states.
For systems where the interpolation deviates significantly from the MEP this estimate can be much too large.
In which case, a maximum estimated value for the barrier can be provided, which then decreases proportionally with $d_i$.

\topic In summary, the method involves iteratively performing the following three steps:
\begin{enumerate}
  \item Reduce the target separation using \cref{eq:diteration}.
  \item Compute the constraint coefficients using \cref{eq:ke,eq:kd}.
  \item Minimise the potential, \cref{eq:bitss}, and recompute the coefficients every 100 iterations of the minimisation.
\end{enumerate}
Any minimisation algorithm can be used in step three; the L-BFGS algorithm is chosen in this work, owing to its fast convergence and low memory requirement for large numbers of degrees of freedom \cite{Liu1989}.
Finally, to determine convergence of the method, a criteria can be based upon either the separation between the states, or the size of gradient midway between the two states.

\topic The typical trajectory of the two states in the BITSS method is shown in \cref{fig:toy2d}a.
Initially, the lower energy state jumps up to satisfy the equal energy constraint and then moves to reduce the separation as much as possible without increasing it's energy.
Then, the two states begin to converge directly towards one another, before being deflected in the direction of a saddle in the potential.
Consequently, if there are multiple possible pathways between two states, BITSS will be biased towards identifying pathways that are more direct or with lower energy.
As the two states approach the transition state the path they begin to closely follow the minimum energy pathway (MEP), which is the path that follows the gradient from the minima and passes through the transition state.
Therefore, the final two states are positioned in the direction of the negative curvature eigenvector, $\bm{\hat{\tau}}$, as shown in \cref{fig:toy2d}c.
So it is also possible to obtain the eigenvector and eigenvalue using BITSS.

\topic Once BITSS has identified the transition state it is possible to find the full MEP.
This is done by tracing the trajectory of downhill minimisations from either side of the transition state, which are provided by the two final states of BITSS.
To ensure this pathway is sufficiently smooth the maximum step size of the minimisation should be restricted.


%\section{Speed Comparison}
\begin{figure*}[htb]
  \centering
  \includegraphics{fig/speedtest-a.pdf}%
  \includegraphics{fig/speedtest-b.pdf}%
  \includegraphics{fig/speedtest-c.pdf}%
  \caption{
    The rate of convergence to the transition state for (a) a Lennard-Jones seven-particle cluster, (b) cylindrical shell buckling, (c) wetting of a chemically-striped surface.
    Above are the two minimum energy states and the transition state between them.
    Below is shown the convergence to the transition state as a function of the number of gradient calculations using the BITSS (black line) and string methods.
    The string method is repeated with a differing number of images along the string, the number of which is listed in the legend in (c).
  }
  \label{fig:speedtest}
\end{figure*}

\topic As a result of the BITSS method using only two states it has the potential to be significantly faster than chain-of-states approaches which typically use tens of states.
This is tested here by measuring the speed at which the BITSS method converges to the transition state and comparing it to the string method and DNEB for several representitive examples.
The accuracy of the BITSS method is calculated using the distance between the correct transition state and the average of the two states.
While for the chain-of-states methods, the accuracy is measured as the distance of the transition state from a cubic spline interpolation of the points along the string.
For each method, the accuracy is measured against the number of times the energy and gradient of the system are evaluated.
The results of this are shown in \cref{fig:speedtest} and \temp{SI} for the string method and DNEB respectively.

\topic The first system to be tested is a two-dimensional seven-particle cluster with Lennard-Jones interactions.
The transition begins with the hexagonal cluster and involves two particles shifting to reach a higher energy configuration.

\topic The second system is an elastic cylindrical shell under an applied strain.
This undergoes a transition from an unbuckled state to a singly-dimpled state, which is significant because it defines the minimum energy required for the cylinder to buckle \cite{Panter2019}.
A 2D triangular mesh is used to model this system, with the following energy function,
\begin{equation}
  E = \sum_i k^S_i (r_i - r^0_i)^2 + \sum_j k^B_j [1 + \text{cos}(\theta_j - \theta^0_j)].
\end{equation}
The first term provides the stretching energy for the system using a sum over all the bonds, where $r_i$ is the length, $r^0_i$ --- the equilibrium length, and $k^S_i$ --- the stretching rigidity of bond $i$.
The second term models the bending energy by representing each pair of adjacent triangles as an elastic hinge.
For hinge $j$, $\theta_j$ is the dihedral angle, $\theta^0_j$ --- the equilibrium angle, and $k^B_j$ --- the bending rigidity.

\topic The final system involves a droplet situated on a chemically striped surface with both hydrophilic and hydrophobic regions.
This has two possible states, with the droplet either upon one hydrophilic stripe, or straddling across two.
This is modelled on a 200x200 2D grid using a phase-field model \cite{Panter2019b}, which has an order parameter, $\phi(\bm{r})$, representing the phase of the liquid ($\phi=-1$ for gas, $\phi=1$ for liquid).
The energy functional has four separate terms,
\begin{equation}
  E[\phi] = E_B + E_I + E_S + E_V.
\end{equation}
The first term, $E_B$, uses a double well potential to set values of $\phi=\pm1$ in the bulk.
The second term then provides the interfacial energy between the liquid and gas by imposing an energy penalty to changes in $\phi$.
$E_S$ is the solid-liquid interaction energy, which sets the contact angles of the hydrophilic and hydrophobic regions to 60\si{\degree} and 110\si{\degree} respectively.
Finally, $E_V$ constrains the volume of the liquid drop by penalising any variation from 11\% of the total volume of the system.

\topic The results for this test, shown in \cref{fig:speedtest}, demonstrate two main advantages for the BITSS method.
Firstly, the slope illustrates that the rate of convergence is indeed much greater for the BITSS method.
It is matched only by the initial rate of the string method using less than ten images, however these soon slow and are relatively inaccurate.
Secondly, the BITSS method can more accurately approximate the transition state by orders of magnitude, particularly for the wetting example, where the string method does not significantly improve upon the initial pathway estimate.
Therefore, unlike the string method, BITSS does not require a second approach to accurately locate the transition state.
Similar results are also found when comparing against against DNEB, which can be seen in the supplementary information.


%\section{Adaptive mesh}
\begin{figure}[htb]
  \centering
  \begin{tabular}[b]{c}
    \includegraphics{fig/changingmesh.pdf}\\
    \includegraphics{fig/differentgrids.pdf}%
  \end{tabular}
  \caption{
    (a) Snapshots of the BITSS method applied to the buckling of a cylinder with a changing mesh.
        The radial displacement relative the unbuckled cylinder is shown, as well as the underlying triangular mesh.
    (b) BITSS applied to the striped wetting example with different resolutions for the two states.
        Each grid cell denotes 50x50 lattice nodes.
        The zoomed axes shows the difference in the fluid interface between the two final states, as well as the approximated transition state (solid black line).
        This is compared to the dashed line showing the true transition state found using a high resolution for both states.
  }
  \label{fig:adaptivemesh}
\end{figure}

\topic To further increase the speed of the method we can use adaptive remeshing and coarse-graining to lower the resolution of the system in regions with few changing features.
However, this introduces additional complexity to the method.
It must be able to handle both the degrees of freedom changing, and the two states having separate discretisations.
Here we test each of these issues individually.

\topic Firstly, we demonstrate in \cref{fig:adaptivemesh}a that BITSS is indeed able to handle the discretisation changing as the method runs.
For this we use the cylindrical buckling example with the resolution increasing from 40 to 100 triangles around the cylinder.
These changes are performed after the minimisation step in each iteration of the BITSS method.
This demonstrates that BITSS is able to converge to the transition state so long as the remeshing is not so drastic that one state is shifted into the basin of another minima.

\topic For the second test, we investigate how BITSS performs when the two states have different meshes.
The only special consideration needed for this is that the distance measure must take into account the different grids.
This is done here by interpolating one state onto the other grid and computing the Euclidean distance.
For some applications it may be sufficient to use simpler distance measures, such as the difference between average values of the system.
The results for the striped wetting example with two different resolutions (shown in \cref{fig:adaptivemesh}b) demonstates that BITSS is successful.
Notably, however, there is a limit to how close the two states can get.
This is because the energy of the transition state is slightly different on each grid, so eventually the energy constraint will fail.
Despite this, the two states are still able to get very close and give a very good approximation to the transition state.
Furthermore, using an adaptive grid model the grids would also be expected to converge, thereby eliminating the issue.


%\section{Zero-curvature modes}
\topic Interaction potentials in many particulate and macroscale systems are short range, and often limited to contact forces.
Thus, there may be many regions in the potential energy landscape that are locally flat.
Single-ended search methods generally only use local information, and so require specialist treatment to deal with these zero modes.
For example, global translation and rotation in free space require analytic expressions for the translation and rotation vectors, which are known.
However, these single-ended search methods are thwarted by local zero modes.

\topic In \cref{fig:flatdiscontinuous}a we demonstrate the BITSS method applied to a 2D potential with regions containing such zero modes (i \& ii).
We see it is able to successfully converge past these flat regions, even with one very close to the transition state (ii).
In these regions there are no driving forces due to the potential and the energy constraint, which use purely local information about the gradient.
However, the distance constraint continues to pull the states together, preventing them from getting stuck.
When only a single state has a zero-gradient mode then the other is likely to slide down the potential slightly, causing a kink in the pathway (iii).
Although, the two states still remain either side of the dividing ridge and so the result will be unaffected.
Indeed, only if the zero-gradient mode is at the transition state in the direction of the pathway will it cause BITSS to fail.
But in this case there is no single transition state, instead the MEP would have a flat section at the top of its energy profile.


\begin{figure}[htb]
  \centering
  \begin{tabular}[b]{c}
    \includegraphics{fig/zeroeigen.pdf}\\
    \includegraphics{fig/discontinuous.pdf}
  \end{tabular}
  \caption{
    (a) Energy profile of the BITSS pathway on a 2D potential with flat regions.
        The minima and transition state are marked by red and magenta dots respectively.
        Points of interest are labelled by i-iii.
        Top inset: A zoomed in view around the transition state.
        Bottom inset: The pathway taken, with the edges of the flat regions marked by dashed lines.
    (b) Disconnectivity graph of the energy landscape for a seven-particle cluster with a discontinuous hard-core pair potential.
    (c) Plot of the discontinuous potential.
        The repulsive section of the usual Lennard-Jones potential is also shown in grey.
  }
  \label{fig:flatdiscontinuous}
\end{figure}


%\section{Gradient-free problems}
\topic In many cases contact forces may be chosen to be infinite to prevent overlap of solid bodies.
This results in the energy landscape becoming discontinuous with regions of undefined gradients.
Alternatively, for some systems the potential energy gradients may be prohibitively expensive to compute, or not available.
In both cases, typical landscapes methods that require continuous, differentiable (often twice-differentiable) optimisation functions will not be appropriate.

\topic BITSS can be used for these systems with undefined gradients with just a couple of alterations.
First, the term using gradients in \cref{eq:kd} can be ignored and $\kappa_d$ can instead be calculated using just the second term.
Next we must use a minimiser that does not require a differentiable optimisation function; for this we use simulated annealing \cite{Kirkpatrick1983}.
This has a chance of randomly jumping one state over the dividing barrier, but we can reduce this probability by setting the initial temperature to $T_0 = \frac{\Delta E}{10}$, and the maximum size of random displacements to $d_\text{max} = d(\bm{x}_1, \bm{x}_2) / 100$.

\topic This has been tested using a Lennard-Jones 7-particle cluster with hard cores, resulting in a discontinuous landscape.
Using the gradient-free approach, BITSS is able to successfully find the transition states, allowing us to plot the disconnectivity graph of the system, as shown in \cref{fig:flatdiscontinuous}b.
In the supplementary information this is compared with the results for the standard Lennard-Jones cluster.
It is worth noting that gradient-based approaches are significantly more efficient, so these should be preferred if possible.


\section{Conclusion}
\topic In conclusion, we have demonstrated a novel method for finding transition states that addresses some problems not currently addressed by existing methods.
The method has advantages over current double-ended methods because it is more efficient, both in terms of memory and computational cost, and does not require an initial estimate for the pathway, useful for systems with non-linear pathways.
Furthermore, we have demonstrated that the method can sucessfully be used with adaptive mesh algorithms, as well as systemd with flat or discontinuous energy landscapes.


\bibliographystyle{plain}
\bibliography{../../library.bib}

\end{document}
