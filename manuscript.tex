\documentclass[twocolumn,10pt]{revtex4}

% Maths
\usepackage{bm}
\usepackage{amsmath}
\newcommand{\abs}[1]{\left| #1 \right|}
\newcommand{\grad}{\bm{\nabla}}
\newcommand{\upDelta}{\mathop{}\!\Delta}

% Text
\usepackage{siunitx}
\usepackage{enumitem}
\usepackage{xcolor}

% Figures
\usepackage{graphicx}
\graphicspath{{fig/}}

% Referencing
\usepackage{cleveref}
\Crefname{figure}{Fig.}{Figs.}


\begin{document}
\title{A fast and memory-efficient transition state search method for complex energy landscapes}
\author{Samuel J. Avis}
\author{Jack R. Panter}
\email[]{j.r.panter@durham.ac.uk}
\author{Halim Kusumaatmaja}
\email[]{halim.kusumaatmaja@durham.ac.uk}
\affiliation{Department of Physics, Durham University, South Road, Durham DH1 3LE, UK}

\begin{abstract}
  Locating transition states is crucial for investigating transition mechanisms in wide-ranging phenomena, from atomistic to macroscale systems.
  Existing methods, however, can struggle in problems with a large number of degrees of freedom, on-the-fly adaptive remeshing and coarse-graining, and energy landscapes that are locally flat or discontinuous.
  To resolve these challenges, we introduce a new double-ended method, the Binary-Image Transition State Search (BITSS).
  It uses just two states that converge to the transition state, resulting in a fast, flexible, and memory-efficient method.
  We demonstate its versatility by applying BITSS to three very different classes of problems: Lennard-Jones clusters, shell buckling, and multiphase phase-field models.
\end{abstract}

\maketitle


\section{Introduction}
Transition states are central to the description of reconfiguration mechanisms for systems in chemistry, condensed matter physics, and engineering.
Historically, many computational methods for locating transition states have grown from an atomistic or particulate perspective.
These have proven to be important tools for understanding, for example, protein folding \cite{Bryngelson1995,Onuchic1997}, biological and industrial catalysis \cite{Boehr2006,Kerns2015,Guo2018a}, quantum tunnelling \cite{Richardson2016,Vaillant2019}, crystallisation \cite{Richard2018}, and cluster formation \cite{Wales1998,Wales2012}.

More recently, it is increasingly being recognised that transition states are useful in mesoscale or macroscale systems.
Here, the minimum energy barriers provide important lower bounds to the energy input required for transitions to occur.
This has been used to understand failure in structural engineering applications \cite{Panter2019,Hutchinson2018}, for the development of super liquid-repellent surfaces \cite{Zhang2014,Panter2019b,Amabili2017}, and investigating locomotion through complex terrain for robotics \cite{Othayoth2020}.
Moreover, it is becoming desirable to tailor elastic deformation transitions to enable technologies such as advanced deployable structures \cite{Filipov2015,Zhai2018}, mechanical sensors and actuators \cite{Bertoldi2017,Truby2016,Chi2022,Bonfanti2020}, and energy absorbers \cite{Shan2015,Giri2021} to name but a few.

Transition state search methods generally fall into two categories, single- and double-ended methods.
Single-ended methods are initialised at a single state and attempt to climb to a nearby saddle point.
Examples include eigenvector following \cite{Cerjan1981}, the dimer method \cite{Heyden2005,Kastner2008,Zhang2016}, and climbing image methods \cite{E2007,Ren2013}.
Double-ended methods can be further subdivided into two groups.
The first utilise a chain of states between two minima which is then minimised to provide an estimate for the full transition pathway in addition to the transition state.
Examples are the string method \cite{E2002,E2007} and doubly-nudged elastic band (DNEB) \cite{Trygubenko2004}.
These methods require an appropriate initial interpolation, which can sometimes be challenging to obtain \cite{Wales2012a}.
The final group are bracketing methods, which involve two states converging to the transition state from either side.
These include the Dewar-Healy-Stewart (DHS) algorithm \cite{Dewar1984}, ridge method \cite{Ionova1993}, the step and slide method \cite{Miron2001}, and the double-ended surface walking method \cite{Zhang2013}.

However, a large range of landscapes prove challenging or impossible to explore via these methods.
One key problem arises from the push towards larger and more complex systems \cite{Trefethen2013,Shalf2020,Alexander2020}, resulting in the need to develop algorithms that are more computationally and memory efficient, and can incorporate optimisation strategies such as on-the-fly adaptive remeshing and coarse-graining.
Chain-of-states methods, however, are not particularly efficient as they require tens of states to gain an accurate approximation, and they have issues with remeshing due to each state having a different number of degrees of freedom.
Meanwhile, single-ended methods are not well suited for identifying specific pathways and can spend a large amount of time searching for undesired transition states.
Another major challenge in studying complex energy landscapes relates to the presence of locally flat or discontinuous regions, such as when considering patchy \cite{McMullen2018,Nguemaha2018,Chen2018b} and hard-body \cite{Richard2018,Santra2018} interactions in atomistic simulations, systems of polymer chains \cite{Mokkonen2016}, or collision constraints for macroscopic objects \cite{Wriggers2006}.
Flat zero-modes in the landscape pose issues for single-ended search methods and some bracketing methods that rely only upon local information.
Specialist treatment can sometimes be used such as in the case of global rotation and translation \cite{Page1988}, but they are thwarted by local zero-modes.
Finally, current methods cannot typically be applied in the case of discontinuous potentials, or if the gradient is prohibitively expensive to compute, because continuous, differentiable optimisation functions are required.

In this work we introduce a new bracketing method, the Binary-Image Transition State Search (BITSS) and, using a range of different applications, we demonstrate that it successfully addresses each of the above challenges.
In addition, we also show that BITSS has an improved reliability in contrast to existing bracketing methods.


\section{Results and discussion}
\subsection{BITSS method}
\begin{figure}[tb]
  \includegraphics[width=\linewidth]{toy2d.pdf}
  \caption{\label{fig:toy2d}
    (a) The trajectories of the two states under the BITSS method (orange line) for a simple 2D potential with two minima (blue dots).
        The transition state is shown by a red dot, and the minimum energy pathway by the dashed line.
    (b) A snapshot of the BITSS minimisation showing the driving forces on each state due to the energy constraint, $\bm{F}_E$, and distance constraint, $\bm{F}_D$, with $E_1 < E_2$ and $\mathrm{d}(\bm{x}_1,\bm{x}_2) < d_i$.
    (c) The final configuration of the BITSS method showing the two states in orange, the transition state in red, and the negative curvature eigenvector, $\bm{\hat{\tau}}$.
  }
\end{figure}

The method begins by first initialising the states, $\bm{x}_1$ and $\bm{x}_2$, in the basins of attraction of different local minima, such as the two blue spots in the 2d potential in \cref{fig:toy2d}a.
These can be set to the minima, but this is not a necessary requirement.
The energies of these two states are then minimised, while constraining their separation.
This is iteratively reduced to zero, such that, at iteration $i$, their separation is
\begin{equation}\label{eq:diteration}
  d_i = (1 - f) d_{i-1},
\end{equation}
with $d_0$ taking the value of the separation between the two initial states.
A reduction factor of $f = 0.5$ is successful for most applications, but this can be made smaller to ensure that the states do not slide off the ridge between the two basins of attraction.
Different metrics may be used to compute this distance, although in this work we simply use the Euclidean distance,
\begin{equation}
  \mathrm{d}(\bm{x}_1, \bm{x}_2) = \sqrt{\sum_i (x_{1,i} - x_{2,i})^2}.
\end{equation}
To further ensure that neither state is pulled over the ridge, a secondary constraint enforces equal energies for the two states.
Using this strategy, the two states will meet at the lowest point on the ridge - the transition state.

The two constraints are implemented using energy penalty terms, which result in driving forces on the two states if the constraints are not met, such as in \cref{fig:toy2d}b.
Including these energy penalty terms gives the total BITSS energy for the pair of states,
\begin{multline}\label{eq:bitss}
  E_\text{BITSS}(\bm{x}_1, \bm{x}_2) = E_1 + E_2
    + \kappa_e \left( E_1 - E_2 \right) ^2 \\
    + \kappa_d \left( \mathrm{d}(\bm{x}_1, \bm{x}_2) - d_i \right) ^2,
\end{multline}
where $E_1$ and $E_2$ are the single-state energies, and $\kappa_e$ and $\kappa_d$ parametrise the strengths of the energy and distance constraints.

In this work the L-BFGS algorithm is chosen to minimise this energy, owing to its fast convergence and low memory requirement for large numbers of degrees of freedom \cite{Liu1989}.
However, any other minimisation method can be used instead.

To ensure that the transition state is located successfully, the constraint strengths $\kappa_d$ and $\kappa_e$ are updated as the algorithm proceeds using information from the system.
These are set such that the driving forces due to the constraints and single-state energies are of similar size.
This prevents the constraints from washing out the underlying potential or causing large jumps that make a state pass over the ridge.
This results in the following equations (see the supplementary information for the derivation),
\begin{gather}
  \kappa_e = \frac {\alpha} {2 E_B},
  \label{eq:ke}
  \\
  \kappa_d = \max \left(
    \frac {\sqrt{\abs{\grad E_1}^2 + \abs{\grad E_2}^2}} {2\sqrt{2} \beta d_i} \; , \;
    \frac{E_B}{\beta d_i^2} \right),
  \label{eq:kd}
\end{gather}
where $\grad E_1$ and $\grad E_2$ are the gradients of the energies of the two states, and $\alpha$ and $\beta$ are parameters with recommended values of $\alpha = 10$ and $\beta = 0.1$.
Here, $E_B$ is an estimation for the current energy barrier; evaluated using the difference between the highest energy along a linear interpolation between the two states and the average energy of the two states.
These constraints are initially calculated at the start of each minimisation, and regularly recalculated throughout (once per 100 iterations is used in this work).

In practice, when numerically minimising, the states will jump about slightly which can result in large gradients perpendicular to the optimal movement direction.
To reduce this effect, the gradients used in \cref{eq:kd} are projected in the direction of the separation between the two states:
\begin{equation}
  \abs{\grad E_n} \approx \frac {\abs{(\bm{x}_1 - \bm{x}_2) \cdot \grad E_n}} {\abs{\bm{x}_1 - \bm{x}_2}}.
\end{equation}

In summary, the method involves iteratively performing the following three steps:
\begin{enumerate}[noitemsep,nolistsep]
  \item Reduce the constrained separation, according to \cref{eq:diteration}.
  \item Minimise the potential of the pair of states, \cref{eq:bitss}.
  \item Recompute the constraint coefficients, $\kappa_e$ and $\kappa_d$, at regular intervals using \cref{eq:ke,eq:kd}.
\end{enumerate}
This process is completed once a suitable convergence criterion is reached.
This can either be based upon the separation between the states, or the size of gradient at the midpoint between them.

Using the BITSS approach, the typical trajectories of the states are demonstrated for a simple 2D potential in \cref{fig:toy2d}a.
Initially, the lower energy state jumps up to satisfy the equal energy constraint and then moves to minimise the separation without increasing its energy.
Then, the two states converge directly towards one another, before being deflected towards the saddle in the ridge.
Consequently, if there are multiple possible pathways between two states, BITSS will be biased towards identifying those that are more direct or with lower energy.
Furthermore, the final two states are positioned either side of the transition state in the direction of the negative curvature eigenvector, $\bm{\hat{\tau}}$ (\cref{fig:toy2d}c).
So, BITSS automatically identifies the `reactive mode' and associated eigenvalue in addition to the transition state.
Once the transition state has been identified, it is possible to find the full minimum energy pathway by tracing the trajectory of downhill minimisations from the two final states, which are either side of the saddle.

In the event that there are intermediate stable states, there will be a chain of multiple transition states between the two minima.
In this case, the equal-energy constraint will not prevent the states from passing over the lower energy transition states, so BITSS should converge to the transition state with the highest energy.
This enables the identification of the overall energy barrier, providing estimates about the overall ease of the transition, or the rate for chemical processes.
However, as demonstrated in the supplementary information, if multiple transition states have very similar energies then a smaller distance reduction factor, $f$, may be necessary to ensure that it does indeed converge to the highest transition state.
If, instead, all of transition states or the full pathway are desired, BITSS can be continually repeated from one of the minima downhill from the located transition state and one of the initial minima until the initial minima are piecewise connected by a full pathway.

\subsection{Other bracketing methods}
\begin{figure}[htb]
  \includegraphics[width=\linewidth]{bracketing.pdf}
  \caption{\label{fig:bracketing}
    Trajectories of the bracketing methods on a hooked potential with a single saddle point.
    The minimum energy pathway is shown by the dashed line.
  }
\end{figure}
The BITSS potential in \cref{eq:bitss} and iterative steps above offer key advantages over existing bracketing methods that also use two states to locate the transition state.
For instance, in the ridge method \cite{Ionova1993}, the two images are initially chosen to bracket the largest energy point on an interpolated path between the two endpoints.
However, this is not guaranteed to be on the ridge containing the transition state, and specialist methods are required to avoid high-energy local maxima, or when the initial path contains multiple candidate maxima.
In another example, the double ended surface walking method \cite{Zhang2013} requires Gaussian bias potentials to be added at each iteration to force two dimers to climb uphill in the landscape.
Although, for high numbers of degrees of freedom and many iterations this becomes computationally expensive.

The two methods most similar to BITSS are the DHS and step and slide methods.
In the the step and slide method, the separation between two images is minimised while their energy is fixed (iteratively increasing the energy up to the transition state).
Conversely in the DHS method, the energy of an image is minimised while the image separation is fixed (iteratively decreasing the separation and changing the frozen image up to the transition state).
However, both of these methods fail if the energy of either image ever exceeds the transition state, as demonstrated for the hooked potential in \cref{fig:bracketing}.
Whereas the BITSS method is successful in this case, showing that it is more reliable when the image pair must descend down a ridge from above.

Furthermore, BITSS has an improved efficiency over these two methods.
In the case of DHS, fixing one state in place and optimising the other means that the amount that the separation is reduced must be much smaller than BITSS to ensure that it does not pass over the ridge.
Meanwhile, using step and slide, it is difficult to obtain a reasonable energy increment when the two states are far from the transition state, also leading to a larger than necessary number of iterations.


\subsection{Speed Comparison}
\begin{figure*}[tb]
  \includegraphics[width=\linewidth]{speedtest.pdf}
  \caption{\label{fig:speedtest}
    Comparisons between BITSS, DNEB, and the string method in the convergence to the transition state as a function of the number of calculations of the energy gradient.
    Three test systems are used: (a) a Lennard-Jones seven-particle cluster, (b) cylindrical shell buckling, and (c) wetting of a chemically-striped surface.
    The configurations shown correspond to the two minimum energy states and the transition state, marked by an asterisk.
    The string method and DNEB are repeated with a differing number of images, as listed in the legend in (c).
  }
\end{figure*}

Although reliable, chain-of-states approaches typically use tens of states.
As a result of the BITSS method using only two states, it has the potential to be significantly faster.
This is tested here by measuring the speed of convergence to the transition state for BITSS in comparison with the string method and DNEB for three diverse systems, exhibiting a broad range of energy landscapes.
Details of each are presented in the methods section.

The first system is a two-dimensional, seven-particle cluster, interacting via a Lennard-Jones pair potential.
This is a frequently used test system for studying transition rates \cite{Wales2002,Passerone2001}.
Here, the \num{14} degrees of freedom are the particle coordinates.
The characteristic transition shown in \cref{fig:speedtest}a sees a particle rearrangement between two close-packed clusters.

The second system is an elastic cylindrical shell, modelled by a triangulated mesh of nodes, which interact via extensional and angular springs.
The \num{35400} degrees of freedom are the node coordinates in three-dimensional space.
The characteristic transition in \cref{fig:speedtest}b shows the formation of a stable dimple from an initially unbuckled cylinder.
This transition is essential to capture to predict mechanical failure under strain \cite{Panter2019,Virot2017}.

The final system involves a droplet situated on a chemically striped surface with both hydrophilic and hydrophobic regions.
Droplet transitions on patterned surfaces such as this are vital to understand as powerful bio-inspired liquid manipulation strategies  \cite{Kusumaatmaja2006,Brown2016}.
In the example shown in \cref{fig:speedtest}c, a droplet transitions from two hydrophilic patches to one patch.
Here, the system is represented by a diffuse-interface model, in which the \num{40000} degrees of freedom are the local fluid compositions at each site of the discretised domain.

Convergence is measured using the distance from the transition state.
For BITSS, the average of the two states is used for this, whereas for the chain-of-states methods, the closest point on a cubic spline interpolation is used.
The speed of this convergence is evaluated through comparison with the number of calculations of the potential gradient for a single state.

The results for this test, shown in \cref{fig:speedtest}, demonstrate two main advantages for the BITSS method.
Firstly, the slope illustrates that the rate of convergence is indeed much greater for the BITSS method.
It is matched only by the initial rates of DNEB and the string method using fewer than ten images, however these soon slow and are relatively inaccurate.
Secondly, the BITSS method can more accurately approximate the transition state by orders of magnitude, particularly for the wetting example where the other methods do not significantly improve upon the initial pathway estimate.
Therefore, unlike these chain-of-states methods, BITSS does not require a second approach to accurately locate the transition state.


\subsection{Adaptive discretisation}\label{sec:adaptive}
\begin{figure}[tb]
  \includegraphics[width=\linewidth]{adaptivemesh.pdf}
  \caption{\label{fig:adaptivemesh}
    (a) Snapshots of the BITSS method applied to the buckling of a cylinder with a changing mesh.
        The radial displacement relative the unbuckled cylinder is shown, as well as the underlying triangular mesh.
    (b) BITSS applied to the striped wetting example with different resolutions for the two states.
        Each grid cell denotes 50x50 lattice nodes.
        The zoomed axis shows the difference in the fluid interface between the two final states, as well as the approximated transition state (solid black line).
        This is compared to the transition state found using a high resolution (dashed line).
  }
\end{figure}

Adaptive remeshing and coarse-graining are widely used techniques that we can utilise to further increase the efficiency of BITSS.
These techniques cause issues for most existing double-ended methods because the coupled states may end up with different degrees of freedom.
However, in BITSS the only direct coupling is in the distance measure, $\mathrm{d}(\bm{x}_1,\bm{x}_2)$, which is relatively easy to adapt.
Here we demonstrate the use of adaptive remeshing by considering two separate issues.

Firstly, we show in \cref{fig:adaptivemesh}a that BITSS is able to handle the discretisation adapting, and the number of degrees of freedom changing, as the method runs.
For this we use the cylindrical buckling example with the resolution increasing from 40 to 100 triangles around the cylinder, corresponding to an increase from \num{1760} to \num{11000} degrees of freedom.
This demonstrates that BITSS is able to converge to the transition state so long as the remeshing is not so significant as to shift a state into the basin of attraction of the other minima.

In the second test, shown in \cref{fig:adaptivemesh}b, we demonstrate the use of different meshes for the two states in the striped wetting example.
In this case, the distance measure is adapted by interpolating one state onto the other mesh and computing the Euclidean distance.
However, for some applications a simpler measure may be sufficient, such as the difference between average values of the system.
Using this approach, BITSS is able to closely approach the transition state.
The precision of this convergence is now limited by the transition state energy differing slightly on each grid, but this effect will be reduced when using an adaptive method or a higher resolution.


\subsection{Complex landscapes}
\begin{figure}[tb]
  \includegraphics[width=\linewidth]{flatdiscontinuous.pdf}
  \caption{\label{fig:flatdiscontinuous}
    (a) Energy profile of the BITSS pathway on a 2D potential with flat regions.
        Blue and red dots denote the minima and transition state, respectively.
        Points of interest are labelled by i--iii (see text).
        Top inset: A zoomed in view around the transition state.
        Bottom inset: The pathway taken, with the edges of the flat regions marked by dashed lines.
    (b) Disconnectivity graph of the energy landscape for a seven-particle cluster with a discontinuous hard-core pair potential (orange), and a standard Lennard-Jones potnetial (gray).
        The two graphs are offset for visibility.
    (c) Plot of the discontinuous potential.
        The repulsive section of the standard Lennard-Jones potential is also shown in grey.
  }
\end{figure}

The final challenges we will address are those related to complex landscapes that prove challenging for previous algorithms.
The first is the presence of flat regions in the landscape.
\Cref{fig:flatdiscontinuous}a shows BITSS applied to a 2D landscape with two such regions (i \& ii) that are flat in the $x$-direction.
We see BITSS is able to successfully converge past these flat regions, even with one very close to the transition state (ii).
In these regions there are no driving forces due to the potential and the energy constraint, which use purely local information about the gradient.
However, the distance constraint continues to pull the states together, preventing them from getting stuck.
When only a single state has a zero-gradient mode then the other is likely to slide down the potential slightly (iii), but the two states still remain either side of the dividing ridge and so the result is unaffected.

Finally, we investigate the application of BITSS to systems with undefined gradients, such as when the landscape is discontinuous.
To account for this, the equations for the coefficients must be adapted to not depend upon the gradients, and a gradient-free minimiser (simulated annealing) is used.
These changes are detailed in the methods section.
This has been tested using a Lennard-Jones 7-particle cluster with hard cores, resulting in a discontinuous landscape.
Using the gradient-free approach, BITSS is able to successfully find the transition states, allowing us to plot the disconnectivity graph of the system, shown in \cref{fig:flatdiscontinuous}b.
Compared with the results for the standard Lennard-Jones cluster, the energies of the minima are largely unchanged, but the energies of the transition states are found to be slightly higher.
This indicates that the particles in the Lennard-Jones cluster cut the corner slightly as they transition, whereas this is not possible using the discontinuous potential, resulting in higher energies.
Despite this gradient-free method being feasible, it is worth noting that a gradient-based approach is significantly more efficient, and so should be preferred if possible.


\section{Conclusion}
Overall, we have developed the binary image transition state search (BITSS) algorithm for the efficient location of transition states in traditionally challenging landscapes.
This has distinct advantages for complex pathways owing to the lack of a required initial pathway estimate, as well as the identification of the transition state that provides the overall energy barrier in multi-step pathways.
Furthermore, the demonstrated speed and memory-efficiency will be key as we move towards simulating systems with increasingly large numbers of degrees of freedom, such as in large biomolecules \cite{Lee2018}, or with increasingly fine resolution, such as in finite element domains \cite{Kolev2021}.

A second source of efficiency in the BITSS method comes from the ability to adaptively change the degrees of freedom as the algorithm proceeds.
We demonstrated how transition states could be found by both increasing the resolution upon convergence, and coupling systems with different discretisations.
The ease of coupling two copies of a system and adaptive remeshing, now leads to the possibility of incorporating BITSS into existing optimisation methods, such as surface evolver \cite{Brakke1992} or finite element methods \cite{Kolev2021}, to provide important energy barrier functionality.

Finally, we showed how BITSS can be used to survey discontinuous energy landscapes, demonstrated for a system of attractive hard-core particles.
This opens up possibilities for studying a broad range of systems previously out of reach of conventional landscape methods, but where transition information is valuable.
These include systems with very short range interactions, such as in colloidal clusters, or hard contact forces, such as in the folding of elastic materials, or locomotion and environmental interaction in robotics.

Throughout, we have demonstrated BITSS as a standalone method for the precise location of transition states.
However, hybrid methods may also be employed to further increase speed and efficiency.
Either BITSS could be used to approximately locate the transition state followed by a single ended method for refinement, or BITSS can be used as a refinement method, complimentary to existing chain of states implementations.
This is scope for possible future work, and could be compared against other existing hybrid methods.

The distance metric between the two BITSS images also proves interesting to analyse further.
One question that emerges is whether transition states can be located by coupling two images through a small number of collective properties, rather than the total distance between all degrees of freedom in the system.
A second question concerns landscapes with multiple competing pathways between states.
In such cases, it may be possible to access transition states different from the most direct one by using a biased distance metric.
Overall, these will enable BITSS to access even more challenging landscapes, and those not yet amenable to traditional landscape exploration techniques.


\section{Methods}

\subsection{Implementing the string and DNEB methods}
In the convergence rate comparison in \cref{fig:speedtest}, the string method and doubly-nudged elastic band method (DNEB) are used, based upon \cite{E2007} and \cite{Trygubenko2004}, respectively.
Each method is initialised with 5, 10, or 20 images, using a linear interpolation between the chosen end states.

In each iteration of the string method, each image is evolved downhill for a number of LBFGS iterations before redistributing the images along a cubic-spline interpolated path.
This number of iterations is varied to obtain the fastest convergence, for which we find 2, 50, and 20 to be optimal for the particle cluster, cylidrical buckling, and striped wetting examples, respectively.

In the case of DNEB, the spring constant between the images is instead optimised.
For the particle cluster system, a spring constant of $1$ is used for the 5 image case, and $0.1$ for 10 and 20 images.
For the cylindrical buckling, the spring constant is $0.01$ in all cases.
And for the striped wetting, $10^{-4}$ is used for 5 and 10 images, and $10^{-3}$ for 20 images.

\subsection{BITSS: Changes for undefined gradients}
A couple of alterations to the method must be made to account for situations where the gradients are unknown.
First, the calculation of $\kappa_d$ in \cref{eq:kd} must be adapted to avoid the use of gradients.
This can be done by simply removing the first term and just using the second term in the equation.
Secondly, L-BFGS can no longer be used because it requires knowledge of the gradients.
We must instead use a minimiser that does not require a differentiable optimisation function, for which we use simulated annealing \cite{Kirkpatrick1983}.
This has a chance of randomly jumping one state over the dividing barrier, but we can reduce this probability by limiting the initial temparature and maximum random displacement.
We typically employ $T_0 = E_B / 10$, and $d_\text{max} = d(\bm{x}_1, \bm{x}_2) / 100$.

\subsection{Adaptive discretisation test details}
Here we provide the details for the interpolations and mapping involved in the two examples demonstrating the feasibility of using an adaptive discretisation method.
For the cylindrical buckling example with a changing mesh, the resolution is refined each time the separation between the two states has halved, and is performed at the end of each iteration of the BITSS method.
This involves the number of triangles around the circumference of the cylinder increasing along the sequence: $40 \rightarrow 60 \rightarrow 80 \rightarrow 100$; with the number of degrees of freedom increasing by: $1760 \rightarrow 3960 \rightarrow 7040 \rightarrow 11000$.
The positions of the nodes on the new grid, $\{\bm{n}_i\}$, are determined by linear interpolation from the previous grid $\{\bm{p}_i\}$, using the positions of the unbuckled meshes, $\{\bm{n'}_i\}$ and $\{\bm{p'}_i\}$.
For each node of the new grid, $\bm{n'}_i$, the triangle that contains it is first identified, which we will denote $\{\bm{p'}_1,\bm{p'}_2,\bm{p'}_3\}$, and the barycentric coordinates of the point are computed, $\{\lambda_1,\lambda_2,\lambda_3\}$.
The new position is then given by $\bm{n}_i = \lambda_1 \bm{p}_1 + \lambda_2 \bm{p}_2 + \lambda_3 \bm{p}_3$.

In the wetting example with different resolutions for the two states, the distance is obtained by first mapping the phase field from the higher resolution grid, $\{\phi_{k,l} | k,l \in \{0,1,\cdots,399\}\}$, to the low resolution grid, $\{\phi'_{i,j} | i,j \in \{0,1,\cdots,199\}\}$.
Because a square grid is used with a resolution ratio of two, the mapping involves averaging each 2x2 block to a single point:
\begin{equation}
  \phi'_{i,j} = \frac{1}{4} \left( \phi_{2i,2j} + \phi_{2i+1,2j} + \phi_{2i,2j+1} + \phi_{2i+1,2j+1} \right).
\end{equation}
Then the separation from the other state, $\{\widetilde{\phi}_{i,j}\}$, is computed using the 2-norm,
\begin{equation}
  d = \sqrt{\sum_{i,j} \left( \phi'_{i,j} - \widetilde{\phi}_{i,j} \right)^2}.
\end{equation}
Finally, the gradient of the distance with respect to each point must be mapped back to the higher-resolution grid, which is done by assigning a quarter of each component back to its the original four points:
\begin{equation}
  \frac{\partial d}{\partial \phi_{k,l}} =
    \frac{\partial \phi'_{i,j}}{\partial \phi_{k,l}} \frac{\partial d}{\partial \phi'_{i,j}} =
    \frac{1}{4} \frac{\partial d}{\partial \phi'_{i,j}} =
    \frac{1}{4} \frac{\phi'_{i,j} - \widetilde{\phi}_{i,j}}{d},
\end{equation}
where $k \in \{2i, 2i+1\}$, and $l \in \{2j, 2j+1\}$.

\subsection{Energy and gradient expressions for the example systems}
\subsubsection{Particle cluster system}
In the particle cluster example system, the Lennard-Jones potential is used for the interaction between each pair of particles.
Therefore, the potential and its gradient for each pair of particles are
\begin{gather}
  E = 4\epsilon \left[ \left(\frac{\sigma}{r}\right)^{12} - \left(\frac{\sigma}{r}\right)^6 \right], \\
  \frac{\partial E}{\partial \bm{x}_1} = -\frac{\partial E}{\partial \bm{x}_2} =
    \frac{24 \epsilon (\bm{x}_2 - \bm{x}_1)}{r^2} \left[ 2 \left(\frac{\sigma}{r}\right)^{12} - \left(\frac{\sigma}{r}\right)^6 \right],
\end{gather}
where $\bm{x}_1$ and $\bm{x}_2$ are the positions of the two particles, $r$ is their separation, $\epsilon$ is the interaction strength, and $\sigma$ is the particle radius.

\subsubsection{Cylindrical buckling system}
A 2D triangular mesh is used to model the cylindrical buckling system, with the ends of the cylinder fixed in place to apply an axial compression of 0.14\%.
The energy of the system is evaluated by treating all bonds in the mesh as an elastic spring to obtain the stretching energy, and all pairs of adjacent triangles to be connected by elastic hinges, providing the bending energy.
Their expressions are given by
\begin{equation}
  E = \sum_i k^S_i (r_i - r^0_i)^2 + \sum_j k^B_j [1 + \cos(\theta_j - \theta^0_j)].
\end{equation}
The first term is the stretching energy, where $r_i$ is the length, $r^0_i$ is the equilibrium length, and $k^S_i$ is the stretching rigidity of bond $i$.
The second term provides the bending energy, where $\theta_j$ is the dihedral angle, $\theta^0_j$ is the equilibrium angle, and $k^B_j$ is the bending rigidity of hinge $j$.

\begin{figure}[htb]
  \includegraphics{fig/BAHschematic.pdf}
  \caption{\label{fig:BAHschematic}
    Schematic of the bar and hinge model showing the relevent parameters for a single hinge element. $h_i$ and $\bm{\hat{n}}_i$ denote the height and unit normal of each triangle respectively.
  }
\end{figure}
The gradient of the energy can be obtained by individually considering the stretching and bending energies of a single bond and hinge.
For simplicity, we will ignore the index for the bond and hinge.
Using the variables shown in the schematic in \cref{fig:BAHschematic}, the gradient of the stretching energy of the bond between $\bm{x}_2$ and $\bm{x}_3$ is given by,
\begin{equation}
  \frac{\partial E^S}{\partial \bm{x}_2} = - \frac{\partial E^S}{\partial \bm{x}_3} =
    2 k^S (r - r^0) (\bm{x_2} - \bm{x_3}).
\end{equation}
The gradients of the bending energy of the hinge are
\begin{align}
  \frac{\partial E^B}{\partial \bm{x}_1} &= k^B \sin(\theta - \theta^0) \frac{\bm{\hat{n}}_a}{h_a}, \\
  \frac{\partial E^B}{\partial \bm{x}_2} &= - k^B \sin(\theta - \theta^0) \left[\frac{\bm{\hat{n}}_a}{h_a} + \frac{\bm{\hat{n}}_a + \bm{\hat{n}}_b}{r}\right], \\
  \frac{\partial E^B}{\partial \bm{x}_3} &= - k^B \sin(\theta - \theta^0) \left[\frac{\bm{\hat{n}}_b}{h_b} + \frac{\bm{\hat{n}}_a + \bm{\hat{n}}_b}{r}\right], \\
  \frac{\partial E^B}{\partial \bm{x}_4} &= k^B \sin(\theta - \theta^0) \frac{\bm{\hat{n}}_b}{h_b}.
\end{align}

\subsubsection{Striped wetting system}
This modelled on a 200x200 2D grid (and 400x400 in \cref{sec:adaptive}) using a phase-field model \cite{Panter2019b}, which has an order parameter, $\phi(\bm{r})$, representing the phase of the liquid ($\phi=-1$ for gas, $\phi=1$ for liquid).
The energy functional has four separate terms,
\begin{equation} \label{eq:phasefield}
  E[\phi] = E^B[\phi] + E^I[\phi] + E^S[\phi] + E^V[\phi].
\end{equation}
The first term, $E^B$, uses a double well potential to set values of $\phi=\pm1$ in the bulk.
The second term then provides the interfacial energy between the liquid and gas by imposing an energy penalty to gradients in $\phi$.
$E^S$ is the solid-liquid interaction energy, which sets the contact angles of the hydrophilic and hydrophobic regions to 60\si{\degree} and 110\si{\degree}, respectively.
Finally, $E^V$ constrains the volume of the liquid drop by penalising any variation from the target volume.

These four sections of the phase-field model are obtained using the following equations,
\begin{align}
  E^B[\phi] &= \sum_i \frac{1}{\epsilon} \left( \frac{{\phi_i}^4}{4} - \frac{{\phi_i}^2}{2} + \frac{1}{4} \right) \upDelta V,\\
  E^I[\phi] &= \sum_i \frac{\epsilon}{2} \abs{\grad \phi_i}^2 \upDelta V,\\
  E^S[\phi] &= \sum_j \sqrt{2}\cos\theta_j \left( \frac{{\phi_j}^3}{6} - \frac{\phi_j}{2} - \frac{1}{3} \right) \upDelta S,\\
  E^V[\phi] &= k_V \left[ \sum_i \frac{\phi_i + 1}{2} \upDelta V - V_0 \right]^2,
\end{align}
where the index $i$ includes all of the nodes, while $j$ represents the nodes along the solid surface.
$\upDelta V$ and $\upDelta S$ are, respectively, the volume and solid surface areas contained by each individual node.
$\epsilon$ is the liquid-gas interface width (set to 2.5 lattice units), $\theta_j$ is the contact-angle with the solid surface, and $V_0$ is the constrained volume of the liquid drop.
The strength of the volume constraint is parametrised by $k_V$ for which we use a value of $10^4$.
The gradients of these terms are given by,
\begin{align}
  \frac{\partial E^B}{\partial \phi_i} &= \frac{1}{\epsilon} \left( {\phi_i}^3 - \phi_i \right) \upDelta V,\\
  \frac{\partial E^I}{\partial \phi_i} &= \frac{\epsilon}{2} \left[
    \frac{\partial \abs{\grad \phi_i}^2}{\partial \phi_i} +
    \sum_k\frac{\partial \abs{\grad \phi_k}^2}{\partial \phi_i} \right] \upDelta V,\\
  \frac{\partial E^S}{\partial \phi_j} &= \sqrt{2}\cos\theta_j \left( \frac{{\phi_j}^2}{2} - \frac{1}{2} \right) \upDelta S,\\
  \frac{\partial E^V}{\partial \phi_i} &= k_V \left[ \sum_{i'} \frac{\phi_{i'} + 1}{2} \upDelta V - V_0 \right] \upDelta V,
\end{align}
where index $k$ denotes neighboring nodes at which the evaluation of the gradient uses $\phi_i$.


\appendix
\section*{}
\begin{acknowledgments}
  S.~J.~A. is supported by a studentship from the Engineering and Physical Sciences Research Council [Grant No. EP/R513039/1].
  H.~K. and J.~R.~P. acknowledge funding from the Engineering and Physical Sciences Research Council [Grant No. EP/V034154/1].
\end{acknowledgments}

%% start bib
% % \bibliography{../../library.bib}
% \bibliography{bibexport.bib}
% \bibliographystyle{naturemag}
%% mid bib
\begin{thebibliography}{10}
\expandafter\ifx\csname url\endcsname\relax
  \def\url#1{\texttt{#1}}\fi
\expandafter\ifx\csname urlprefix\endcsname\relax\def\urlprefix{URL }\fi
\providecommand{\bibinfo}[2]{#2}
\providecommand{\eprint}[2][]{\url{#2}}

\bibitem{Bryngelson1995}
\bibinfo{author}{Bryngelson, J.~D.}, \bibinfo{author}{Onuchic, J.~N.},
  \bibinfo{author}{Socci, N.~D.} \& \bibinfo{author}{Wolynes, P.~G.}
\newblock \bibinfo{title}{{Funnels, pathways, and the energy landscape of
  protein folding: A synthesis}}.
\newblock \emph{\bibinfo{journal}{Proteins: Structure, Function, and
  Bioinformatics}} \textbf{\bibinfo{volume}{21}}, \bibinfo{pages}{167--195}
  (\bibinfo{year}{1995}).

\bibitem{Onuchic1997}
\bibinfo{author}{Onuchic, J.~N.}, \bibinfo{author}{Luthey-Schulten, Z.} \&
  \bibinfo{author}{Wolynes, P.~G.}
\newblock \bibinfo{title}{{Theory of Protein Folding: The Energy Landscape
  Perspective}}.
\newblock \emph{\bibinfo{journal}{Annual Review of Physical Chemistry}}
  \textbf{\bibinfo{volume}{48}}, \bibinfo{pages}{545--600}
  (\bibinfo{year}{1997}).

\bibitem{Boehr2006}
\bibinfo{author}{Boehr, D.~D.}, \bibinfo{author}{McElheny, D.},
  \bibinfo{author}{Dyson, H.~J.} \& \bibinfo{author}{Wrightt, P.~E.}
\newblock \bibinfo{title}{{The dynamic energy landscape of dihydrofolate
  reductase catalysis}}.
\newblock \emph{\bibinfo{journal}{Science}} \textbf{\bibinfo{volume}{313}},
  \bibinfo{pages}{1638--1642} (\bibinfo{year}{2006}).

\bibitem{Kerns2015}
\bibinfo{author}{Kerns, S.~J.} \emph{et~al.}
\newblock \bibinfo{title}{{The energy landscape of adenylate kinase during
  catalysis}}.
\newblock \emph{\bibinfo{journal}{Nature Structural and Molecular Biology}}
  \textbf{\bibinfo{volume}{22}}, \bibinfo{pages}{124--131}
  (\bibinfo{year}{2015}).

\bibitem{Guo2018a}
\bibinfo{author}{Guo, C.}, \bibinfo{author}{Wang, Z.}, \bibinfo{author}{Wang,
  D.}, \bibinfo{author}{Wang, H.~F.} \& \bibinfo{author}{Hu, P.}
\newblock \bibinfo{title}{{First-Principles Determination of CO Adsorption and
  Desorption on Pt(111) in the Free Energy Landscape}}.
\newblock \emph{\bibinfo{journal}{Journal of Physical Chemistry C}}
  \textbf{\bibinfo{volume}{122}}, \bibinfo{pages}{21478--21483}
  (\bibinfo{year}{2018}).

\bibitem{Richardson2016}
\bibinfo{author}{Richardson, J.~O.} \emph{et~al.}
\newblock \bibinfo{title}{{Concerted hydrogen-bond breaking by quantum
  tunneling in the water hexamer prism}}.
\newblock \emph{\bibinfo{journal}{Science}} \textbf{\bibinfo{volume}{351}},
  \bibinfo{pages}{1310} (\bibinfo{year}{2016}).

\bibitem{Vaillant2019}
\bibinfo{author}{Vaillant, C.~L.}, \bibinfo{author}{Althorpe, S.~C.} \&
  \bibinfo{author}{Wales, D.~J.}
\newblock \bibinfo{title}{{Path Integral Energy Landscapes for Water
  Clusters}}.
\newblock \emph{\bibinfo{journal}{Journal of Chemical Theory and Computation}}
  \textbf{\bibinfo{volume}{15}}, \bibinfo{pages}{33--42}
  (\bibinfo{year}{2019}).

\bibitem{Richard2018}
\bibinfo{author}{Richard, D.} \& \bibinfo{author}{Speck, T.}
\newblock \bibinfo{title}{{Crystallization of hard spheres revisited. I.
  Extracting kinetics and free energy landscape from forward flux sampling}}.
\newblock \emph{\bibinfo{journal}{The Journal of Chemical Physics}}
  \textbf{\bibinfo{volume}{148}}, \bibinfo{pages}{124110}
  (\bibinfo{year}{2018}).

\bibitem{Wales1998}
\bibinfo{author}{Wales, D.~J.}, \bibinfo{author}{Miller, M.~A.} \&
  \bibinfo{author}{Walsh, T.~R.}
\newblock \bibinfo{title}{{Archetypal energy landscapes}}.
\newblock \emph{\bibinfo{journal}{Nature}} \textbf{\bibinfo{volume}{394}},
  \bibinfo{pages}{758--760} (\bibinfo{year}{1998}).

\bibitem{Wales2012}
\bibinfo{author}{Wales, D.~J.}
\newblock \bibinfo{title}{{Decoding the energy landscape: Extracting structure,
  dynamics and thermodynamics}}.
\newblock \emph{\bibinfo{journal}{Philosophical Transactions of the Royal
  Society A: Mathematical, Physical and Engineering Sciences}}
  \textbf{\bibinfo{volume}{370}}, \bibinfo{pages}{2877--2899}
  (\bibinfo{year}{2012}).

\bibitem{Panter2019}
\bibinfo{author}{Panter, J.~R.}, \bibinfo{author}{Chen, J.},
  \bibinfo{author}{Zhang, T.} \& \bibinfo{author}{Kusumaatmaja, H.}
\newblock \bibinfo{title}{{Harnessing energy landscape exploration to control
  the buckling of cylindrical shells}}.
\newblock \emph{\bibinfo{journal}{Communications Physics}}
  \textbf{\bibinfo{volume}{2}}, \bibinfo{pages}{151} (\bibinfo{year}{2019}).

\bibitem{Hutchinson2018}
\bibinfo{author}{Hutchinson, J.~W.} \& \bibinfo{author}{Thompson, J. M.~T.}
\newblock \bibinfo{title}{{Imperfections and energy barriers in shell
  buckling}}.
\newblock \emph{\bibinfo{journal}{International Journal of Solids and
  Structures}} \textbf{\bibinfo{volume}{148-149}}, \bibinfo{pages}{157--168}
  (\bibinfo{year}{2018}).

\bibitem{Zhang2014}
\bibinfo{author}{Zhang, Y.} \& \bibinfo{author}{Ren, W.}
\newblock \bibinfo{title}{{Numerical study of the effects of surface topography
  and chemistry on the wetting transition using the string method}}.
\newblock \emph{\bibinfo{journal}{The Journal of Chemical Physics}}
  \textbf{\bibinfo{volume}{141}}, \bibinfo{pages}{244705}
  (\bibinfo{year}{2014}).

\bibitem{Panter2019b}
\bibinfo{author}{Panter, J.~R.}, \bibinfo{author}{Gizaw, Y.} \&
  \bibinfo{author}{Kusumaatmaja, H.}
\newblock \bibinfo{title}{{Multifaceted design optimization for superomniphobic
  surfaces}}.
\newblock \emph{\bibinfo{journal}{Science Advances}}
  \textbf{\bibinfo{volume}{5}}, \bibinfo{pages}{eaav7328}
  (\bibinfo{year}{2019}).

\bibitem{Amabili2017}
\bibinfo{author}{Amabili, M.}, \bibinfo{author}{Giacomello, A.},
  \bibinfo{author}{Meloni, S.} \& \bibinfo{author}{Casciola, C.~M.}
\newblock \bibinfo{title}{{Collapse of superhydrophobicity on nanopillared
  surfaces}}.
\newblock \emph{\bibinfo{journal}{Physical Review Fluids}}
  \textbf{\bibinfo{volume}{2}}, \bibinfo{pages}{034202} (\bibinfo{year}{2017}).

\bibitem{Othayoth2020}
\bibinfo{author}{Othayoth, R.}, \bibinfo{author}{Thoms, G.} \&
  \bibinfo{author}{Li, C.}
\newblock \bibinfo{title}{{An energy landscape approach to locomotor
  transitions in complex 3D terrain}}.
\newblock \emph{\bibinfo{journal}{Proceedings of the National Academy of
  Sciences}} \textbf{\bibinfo{volume}{117}}, \bibinfo{pages}{14987--14995}
  (\bibinfo{year}{2020}).

\bibitem{Filipov2015}
\bibinfo{author}{Filipov, E.~T.}, \bibinfo{author}{Tachi, T.},
  \bibinfo{author}{Paulino, G.~H.} \& \bibinfo{author}{Weitz, D.~A.}
\newblock \bibinfo{title}{{Origami tubes assembled into stiff, yet
  reconfigurable structures and metamaterials}}.
\newblock \emph{\bibinfo{journal}{Proceedings of the National Academy of
  Sciences of the United States of America}} \textbf{\bibinfo{volume}{112}},
  \bibinfo{pages}{12321} (\bibinfo{year}{2015}).

\bibitem{Zhai2018}
\bibinfo{author}{Zhai, Z.}, \bibinfo{author}{Wang, Y.} \&
  \bibinfo{author}{Jiang, H.}
\newblock \bibinfo{title}{{Origami-inspired, on-demand deployable and
  collapsible mechanical metamaterials with tunable stiffness}}.
\newblock \emph{\bibinfo{journal}{Proceedings of the National Academy of
  Sciences of the United States of America}} \textbf{\bibinfo{volume}{115}},
  \bibinfo{pages}{2032--2037} (\bibinfo{year}{2018}).

\bibitem{Bertoldi2017}
\bibinfo{author}{Bertoldi, K.}, \bibinfo{author}{Vitelli, V.},
  \bibinfo{author}{Christensen, J.} \& \bibinfo{author}{{Van Hecke}, M.}
\newblock \bibinfo{title}{{Flexible mechanical metamaterials}}.
\newblock \emph{\bibinfo{journal}{Nature Reviews Materials}}
  \textbf{\bibinfo{volume}{2}}, \bibinfo{pages}{17066} (\bibinfo{year}{2017}).

\bibitem{Truby2016}
\bibinfo{author}{Truby, R.~L.} \& \bibinfo{author}{Lewis, J.~A.}
\newblock \bibinfo{title}{{Printing soft matter in three dimensions}}.
\newblock \emph{\bibinfo{journal}{Nature}} \textbf{\bibinfo{volume}{540}},
  \bibinfo{pages}{371--378} (\bibinfo{year}{2016}).

\bibitem{Chi2022}
\bibinfo{author}{Chi, Y.} \emph{et~al.}
\newblock \bibinfo{title}{{Bistable and Multistable Actuators for Soft Robots:
  Structures, Materials, and Functionalities}}.
\newblock \emph{\bibinfo{journal}{Advanced Materials}}  (\bibinfo{year}{2022}).

\bibitem{Bonfanti2020}
\bibinfo{author}{Bonfanti, S.}, \bibinfo{author}{Guerra, R.},
  \bibinfo{author}{Font-Clos, F.}, \bibinfo{author}{Rayneau-Kirkhope, D.} \&
  \bibinfo{author}{Zapperi, S.}
\newblock \bibinfo{title}{{Automatic design of mechanical metamaterial
  actuators}}.
\newblock \emph{\bibinfo{journal}{Nature Communications}}
  \textbf{\bibinfo{volume}{11}}, \bibinfo{pages}{4162} (\bibinfo{year}{2020}).

\bibitem{Shan2015}
\bibinfo{author}{Shan, S.} \emph{et~al.}
\newblock \bibinfo{title}{{Multistable Architected Materials for Trapping
  Elastic Strain Energy}}.
\newblock \emph{\bibinfo{journal}{Advanced Materials}}
  \textbf{\bibinfo{volume}{27}}, \bibinfo{pages}{4296--4301}
  (\bibinfo{year}{2015}).

\bibitem{Giri2021}
\bibinfo{author}{Giri, T.~R.} \& \bibinfo{author}{Mailen, R.}
\newblock \bibinfo{title}{{Controlled snapping sequence and energy absorption
  in multistable mechanical metamaterial cylinders}}.
\newblock \emph{\bibinfo{journal}{International Journal of Mechanical
  Sciences}} \textbf{\bibinfo{volume}{204}}, \bibinfo{pages}{106541}
  (\bibinfo{year}{2021}).

\bibitem{Cerjan1981}
\bibinfo{author}{Cerjan, C.~J.} \& \bibinfo{author}{Miller, W.~H.}
\newblock \bibinfo{title}{{On finding transition states}}.
\newblock \emph{\bibinfo{journal}{The Journal of Chemical Physics}}
  \textbf{\bibinfo{volume}{75}}, \bibinfo{pages}{2800--2806}
  (\bibinfo{year}{1981}).

\bibitem{Heyden2005}
\bibinfo{author}{Heyden, A.}, \bibinfo{author}{Bell, A.~T.} \&
  \bibinfo{author}{Keil, F.~J.}
\newblock \bibinfo{title}{{Efficient methods for finding transition states in
  chemical reactions: Comparison of improved dimer method and partitioned
  rational function optimization method}}.
\newblock \emph{\bibinfo{journal}{Journal of Chemical Physics}}
  \textbf{\bibinfo{volume}{123}}, \bibinfo{pages}{224101}
  (\bibinfo{year}{2005}).

\bibitem{Kastner2008}
\bibinfo{author}{K{\"{a}}stner, J.} \& \bibinfo{author}{Sherwood, P.}
\newblock \bibinfo{title}{{Superlinearly converging dimer method for transition
  state search}}.
\newblock \emph{\bibinfo{journal}{Journal of Chemical Physics}}
  \textbf{\bibinfo{volume}{128}}, \bibinfo{pages}{014106}
  (\bibinfo{year}{2008}).

\bibitem{Zhang2016}
\bibinfo{author}{Zhang, L.}, \bibinfo{author}{Du, Q.} \&
  \bibinfo{author}{Zheng, Z.}
\newblock \bibinfo{title}{{Optimization-based Shrinking Dimer Method for
  Finding Transition States}}.
\newblock \emph{\bibinfo{journal}{SIAM Journal on Scientific Computing}}
  \textbf{\bibinfo{volume}{38}}, \bibinfo{pages}{A528--A544}
  (\bibinfo{year}{2016}).

\bibitem{E2007}
\bibinfo{author}{E, W.}, \bibinfo{author}{Ren, W.} \&
  \bibinfo{author}{Vanden-Eijnden, E.}
\newblock \bibinfo{title}{{Simplified and improved string method for computing
  the minimum energy paths in barrier-crossing events}}.
\newblock \emph{\bibinfo{journal}{The Journal of Chemical Physics}}
  \textbf{\bibinfo{volume}{126}}, \bibinfo{pages}{164103}
  (\bibinfo{year}{2007}).

\bibitem{Ren2013}
\bibinfo{author}{Ren, W.} \& \bibinfo{author}{Vanden-Eijnden, E.}
\newblock \bibinfo{title}{{A climbing string method for saddle point search}}.
\newblock \emph{\bibinfo{journal}{Journal of Chemical Physics}}
  \textbf{\bibinfo{volume}{138}}, \bibinfo{pages}{134105}
  (\bibinfo{year}{2013}).

\bibitem{E2002}
\bibinfo{author}{E, W.}, \bibinfo{author}{Ren, W.} \&
  \bibinfo{author}{Vanden-Eijnden, E.}
\newblock \bibinfo{title}{{String method for the study of rare events}}.
\newblock \emph{\bibinfo{journal}{Physical Review B}}
  \textbf{\bibinfo{volume}{66}}, \bibinfo{pages}{052301}
  (\bibinfo{year}{2002}).

\bibitem{Trygubenko2004}
\bibinfo{author}{Trygubenko, S.~A.} \& \bibinfo{author}{Wales, D.~J.}
\newblock \bibinfo{title}{{A doubly nudged elastic band method for finding
  transition states}}.
\newblock \emph{\bibinfo{journal}{Journal of Chemical Physics}}
  \textbf{\bibinfo{volume}{120}}, \bibinfo{pages}{2082--2094}
  (\bibinfo{year}{2004}).

\bibitem{Wales2012a}
\bibinfo{author}{Wales, D.~J.} \& \bibinfo{author}{Carr, J.~M.}
\newblock \bibinfo{title}{{Quasi-Continuous Interpolation Scheme for Pathways
  between Distant Configurations}}.
\newblock \emph{\bibinfo{journal}{Journal of Chemical Theory and Computation}}
  \textbf{\bibinfo{volume}{8}}, \bibinfo{pages}{5020--5034}
  (\bibinfo{year}{2012}).

\bibitem{Trefethen2013}
\bibinfo{author}{Trefethen, A.~E.} \& \bibinfo{author}{Thiyagalingam, J.}
\newblock \bibinfo{title}{{Energy-aware software: Challenges, opportunities and
  strategies}}.
\newblock \emph{\bibinfo{journal}{Journal of Computational Science}}
  \textbf{\bibinfo{volume}{4}}, \bibinfo{pages}{444--449}
  (\bibinfo{year}{2013}).

\bibitem{Shalf2020}
\bibinfo{author}{Shalf, J.}
\newblock \bibinfo{title}{{The future of computing beyond Moore's Law}}.
\newblock \emph{\bibinfo{journal}{Philosophical Transactions Royal Society}}
  \textbf{\bibinfo{volume}{378}}, \bibinfo{pages}{20190061}
  (\bibinfo{year}{2020}).

\bibitem{Alexander2020}
\bibinfo{author}{Alexander, F.} \emph{et~al.}
\newblock \bibinfo{title}{{Exascale applications: Skin in the game}}.
\newblock \emph{\bibinfo{journal}{Philosophical Transactions of the Royal
  Society A}} \textbf{\bibinfo{volume}{378}}, \bibinfo{pages}{20190056}
  (\bibinfo{year}{2020}).

\bibitem{McMullen2018}
\bibinfo{author}{McMullen, A.}, \bibinfo{author}{Holmes-Cerfon, M.},
  \bibinfo{author}{Sciortino, F.}, \bibinfo{author}{Grosberg, A.~Y.} \&
  \bibinfo{author}{Brujic, J.}
\newblock \bibinfo{title}{{Freely Jointed Polymers Made of Droplets}}.
\newblock \emph{\bibinfo{journal}{Physical Review Letters}}
  \textbf{\bibinfo{volume}{121}}, \bibinfo{pages}{138002}
  (\bibinfo{year}{2018}).

\bibitem{Nguemaha2018}
\bibinfo{author}{Nguemaha, V.} \& \bibinfo{author}{Zhou, H.~X.}
\newblock \bibinfo{title}{{Liquid-Liquid Phase Separation of Patchy Particles
  Illuminates Diverse Effects of Regulatory Components on Protein Droplet
  Formation}}.
\newblock \emph{\bibinfo{journal}{Scientific Reports}}
  \textbf{\bibinfo{volume}{8}}, \bibinfo{pages}{1--11} (\bibinfo{year}{2018}).

\bibitem{Chen2018b}
\bibinfo{author}{Chen, D.}, \bibinfo{author}{Zhang, G.} \&
  \bibinfo{author}{Torquato, S.}
\newblock \bibinfo{title}{{Inverse Design of Colloidal Crystals via Optimized
  Patchy Interactions}}.
\newblock \emph{\bibinfo{journal}{Journal of Physical Chemistry B}}
  \textbf{\bibinfo{volume}{122}}, \bibinfo{pages}{8462--8468}
  (\bibinfo{year}{2018}).

\bibitem{Santra2018}
\bibinfo{author}{Santra, M.}, \bibinfo{author}{Singh, R.~S.} \&
  \bibinfo{author}{Bagchi, B.}
\newblock \bibinfo{title}{{Polymorph selection during crystallization of a
  model colloidal fluid with a free energy landscape containing a metastable
  solid}}.
\newblock \emph{\bibinfo{journal}{Physical Review E}}
  \textbf{\bibinfo{volume}{98}}, \bibinfo{pages}{032606}
  (\bibinfo{year}{2018}).

\bibitem{Mokkonen2016}
\bibinfo{author}{M{\"{o}}kk{\"{o}}nen, H.}, \bibinfo{author}{Ala-Nissila, T.}
  \& \bibinfo{author}{J{\'{o}}nsson, H.}
\newblock \bibinfo{title}{{Efficient dynamical correction of the transition
  state theory rate estimate for a flat energy barrier}}.
\newblock \emph{\bibinfo{journal}{The Journal of Chemical Physics}}
  \textbf{\bibinfo{volume}{145}}, \bibinfo{pages}{094901}
  (\bibinfo{year}{2016}).

\bibitem{Wriggers2006}
\bibinfo{author}{Wriggers, P.}
\newblock \emph{\bibinfo{title}{{Computational Contact Mechanics}}}
  (\bibinfo{publisher}{Springer Berlin Heidelberg}, \bibinfo{address}{Berlin,
  Heidelberg}, \bibinfo{year}{2006}).

\bibitem{Page1988}
\bibinfo{author}{Page, M.} \& \bibinfo{author}{McIver, J.~W.}
\newblock \bibinfo{title}{{On evaluating the reaction path Hamiltonian}}.
\newblock \emph{\bibinfo{journal}{The Journal of Chemical Physics}}
  \textbf{\bibinfo{volume}{88}}, \bibinfo{pages}{922--935}
  (\bibinfo{year}{1988}).

\bibitem{Liu1989}
\bibinfo{author}{Liu, D.~C.} \& \bibinfo{author}{Nocedal, J.}
\newblock \bibinfo{title}{{On the limited memory bfgs method for large scale
  optimization}}.
\newblock \emph{\bibinfo{journal}{Mathematical Programming}}
  \textbf{\bibinfo{volume}{45}}, \bibinfo{pages}{503--528}
  (\bibinfo{year}{1989}).

\bibitem{Miron2001}
\bibinfo{author}{Miron, R.~A.} \& \bibinfo{author}{Fichthorn, K.~A.}
\newblock \bibinfo{title}{{The Step and Slide method for finding saddle points
  on multidimensional potential surfaces}}.
\newblock \emph{\bibinfo{journal}{The Journal of Chemical Physics}}
  \textbf{\bibinfo{volume}{115}}, \bibinfo{pages}{8742--8747}
  (\bibinfo{year}{2001}).

\bibitem{Dewar1984}
\bibinfo{author}{Dewar, M. J.~S.}, \bibinfo{author}{Healy, E.~F.} \&
  \bibinfo{author}{Stewart, J. J.~P.}
\newblock \bibinfo{title}{{Location of transition states in reaction
  mechanisms}}.
\newblock \emph{\bibinfo{journal}{Journal of the Chemical Society, Faraday
  Transactions 2}} \textbf{\bibinfo{volume}{80}}, \bibinfo{pages}{227}
  (\bibinfo{year}{1984}).

\bibitem{Ionova1993}
\bibinfo{author}{Ionova, I.~V.} \& \bibinfo{author}{Carter, E.~A.}
\newblock \bibinfo{title}{{Ridge method for finding saddle points on potential
  energy surfaces}}.
\newblock \emph{\bibinfo{journal}{The Journal of Chemical Physics}}
  \textbf{\bibinfo{volume}{98}}, \bibinfo{pages}{6377--6386}
  (\bibinfo{year}{1993}).

\bibitem{Zhang2013}
\bibinfo{author}{Zhang, X.-J.}, \bibinfo{author}{Shang, C.} \&
  \bibinfo{author}{Liu, Z.-P.}
\newblock \bibinfo{title}{{Double-Ended Surface Walking Method for Pathway
  Building and Transition State Location of Complex Reactions}}.
\newblock \emph{\bibinfo{journal}{Journal of Chemical Theory and Computation}}
  \textbf{\bibinfo{volume}{9}}, \bibinfo{pages}{5745--5753}
  (\bibinfo{year}{2013}).

\bibitem{Wales2002}
\bibinfo{author}{Wales, D.~J.}
\newblock \bibinfo{title}{{Discrete path sampling}}.
\newblock \emph{\bibinfo{journal}{Molecular Physics}}
  \textbf{\bibinfo{volume}{100}}, \bibinfo{pages}{3285--3305}
  (\bibinfo{year}{2002}).

\bibitem{Passerone2001}
\bibinfo{author}{Passerone, D.} \& \bibinfo{author}{Parrinello, M.}
\newblock \bibinfo{title}{{Action-Derived Molecular Dynamics in the Study of
  Rare Events}}.
\newblock \emph{\bibinfo{journal}{Physical Review Letters}}
  \textbf{\bibinfo{volume}{87}}, \bibinfo{pages}{108302}
  (\bibinfo{year}{2001}).

\bibitem{Virot2017}
\bibinfo{author}{Virot, E.}, \bibinfo{author}{Kreilos, T.},
  \bibinfo{author}{Schneider, T.~M.} \& \bibinfo{author}{Rubinstein, S.~M.}
\newblock \bibinfo{title}{{Stability Landscape of Shell Buckling}}.
\newblock \emph{\bibinfo{journal}{Physical Review Letters}}
  \textbf{\bibinfo{volume}{119}}, \bibinfo{pages}{224101}
  (\bibinfo{year}{2017}).

\bibitem{Kusumaatmaja2006}
\bibinfo{author}{Kusumaatmaja, H.}, \bibinfo{author}{L{\'{e}}opold{\`{e}}s,
  J.}, \bibinfo{author}{Dupuis, A.} \& \bibinfo{author}{Yeomans, J.~M.}
\newblock \bibinfo{title}{{Drop dynamics on chemically patterned surfaces}}.
\newblock \emph{\bibinfo{journal}{Europhysics Letters}}
  \textbf{\bibinfo{volume}{73}}, \bibinfo{pages}{740--746}
  (\bibinfo{year}{2006}).

\bibitem{Brown2016}
\bibinfo{author}{Brown, P.~S.} \& \bibinfo{author}{Bhushan, B.}
\newblock \bibinfo{title}{{Bioinspired materials for water supply and
  management: water collection, water purification and separation of water from
  oil}}.
\newblock \emph{\bibinfo{journal}{Philosophical Transactions of the Royal
  Society A}} \textbf{\bibinfo{volume}{374}}, \bibinfo{pages}{20160135}
  (\bibinfo{year}{2016}).

\bibitem{Lee2018}
\bibinfo{author}{Lee, C.~T.} \& \bibinfo{author}{Amaro, R.~E.}
\newblock \bibinfo{title}{{Exascale Computing: A New Dawn for Computational
  Biology}}.
\newblock \emph{\bibinfo{journal}{Computing in Science {\&} Engineering}}
  \textbf{\bibinfo{volume}{20}}, \bibinfo{pages}{18--25}
  (\bibinfo{year}{2018}).

\bibitem{Kolev2021}
\bibinfo{author}{Kolev, T.} \emph{et~al.}
\newblock \bibinfo{title}{{Efficient exascale discretizations: High-order
  finite element methods}}.
\newblock \emph{\bibinfo{journal}{The International Journal of High Performance
  Computing Applications}} \textbf{\bibinfo{volume}{35}},
  \bibinfo{pages}{527--552} (\bibinfo{year}{2021}).

\bibitem{Brakke1992}
\bibinfo{author}{Brakke, K.~A.}
\newblock \bibinfo{title}{{The Surface Evolver}}.
\newblock \emph{\bibinfo{journal}{Experimental Mathematics}}
  \textbf{\bibinfo{volume}{1}}, \bibinfo{pages}{141--165}
  (\bibinfo{year}{1992}).

\bibitem{Kirkpatrick1983}
\bibinfo{author}{Kirkpatrick, S.}, \bibinfo{author}{Gelatt, C.~D.} \&
  \bibinfo{author}{Vecchi, M.~P.}
\newblock \bibinfo{title}{{Optimization By Simulated Annealing}}.
\newblock \emph{\bibinfo{journal}{Science}} \textbf{\bibinfo{volume}{220}},
  \bibinfo{pages}{671--680} (\bibinfo{year}{1983}).

\end{thebibliography}
%% end bib

\end{document}
