\documentclass[aps,prl]{revtex4}

% Maths
\usepackage{bm}
\usepackage{amsmath}
\newcommand{\abs}[1]{\left| #1 \right|}
\newcommand{\grad}{\bm{\nabla}}
\newcommand{\upDelta}{\mathop{}\!\Delta}

% Referencing
\usepackage{cleveref}
\renewcommand{\theequation}{S\arabic{equation}}
\renewcommand{\thefigure}{S\arabic{figure}}

% Figures
\usepackage{graphicx}


\begin{document}
\title{Binary Image Transition State Search \\ Supplementary Information}
\author{Samuel J. Avis}
\author{Jack R. Panter}
% \email[]{j.r.panter@durham.ac.uk}
\author{Halim Kusumaatmaja}
% \email[]{halim.kusumaatmaja@durham.ac.uk}
\affiliation{Department of Physics, Durham University, South Road, Durham DH1 3LE, UK}
\maketitle

\section{Constraint coefficients}
The constraint coefficients are chosen to set comparable sizes for each term in the gradient of the BITSS potential,
\begin{equation} \label{eq:gradBITSS}
  \grad E_\text{BITSS} = \sum_{\substack{i=1,2 \\ j\neq i}} \left[ 1 + 2 \kappa_e (E_i - E_j) \right] \grad E_i + 2 \kappa_d (d - d_0) \grad d.
\end{equation}

To obtain the expression for the energy coefficient, $\kappa_e$, we first assume that the separation is fixed, so the distance term can be ignored.
The coefficient $\kappa_e$ must be high enough to prevent one state from being pulled over the ridge, for which the greatest risk occurs when the gradient on one state is much greater than the other, $\abs{\grad E_2} \gg \abs{\grad E_1}$.
In this case the total gradient is approximated by
\begin{equation}
  \grad E_{tot} = \left[ 1 - 2 \kappa_e (E_1 - E_2) \right] \grad E_2.
\end{equation}
This will converge when the term in the square brackets is zero, resulting in $E_1 - E_2 = 1 / 2 \kappa_e$.
This energy difference should be less than current energy barrier, so we can substitute it with $\upDelta E / \alpha$, where $\upDelta E$ is the energy barrier estimate and $\alpha$ is a constant greater than one.
Leaving us with the expression
\begin{equation}
  \kappa_e = \frac {\alpha} {2 \upDelta E}.
\end{equation}

The distance coefficient is determined by assuming that the energies are equal and thus the energy constraint can be ignored.
In this case, convergence will occur when $\grad E_1 + \grad E_2 = -2 \kappa_d (d - d_0) \grad d$.
It is then possible to find the value of $\kappa_d$ for which the magnitude of each side of this equation is equal for a desired relative error in the distance, $\beta = (d - d_0) / d_0$.
(Note: The gradient of the distance with respect to a single state has a magnitude of 1, thus the magnitude of the gradient for the pair of states is $\abs{\grad d} = \sqrt{1^2+1^2} = \sqrt{2}$.)
A lower bound on the size of the coefficient is also set using the estimated energy barrier to ensure that the distance constraint is applied even when the gradient is close to zero, such as when the states are initialised at the minima.
\begin{equation}
  \kappa_d = \mathrm{max} \! \left(
  \frac {\sqrt{\abs{\grad E_1}^2 + \abs{\grad E_2}^2}} {2 \sqrt{2} \beta d_0}, \quad
  \frac{\upDelta E}{\beta d_0^2} \right).
\end{equation}

\begin{figure}[b]
  \includegraphics{fig/paramtest-a.pdf}%
  \includegraphics{fig/paramtest-b.pdf}%
  \caption{\label{fig:paramtest}
    Speed of convergence of the BITSS method under different choices of parameters for (a) the seven-particle cluster, (b) cylindrical buckling.
    The speed is given by the number of evaluations of the gradient until the two states are separated by less than a thousandth of the initial separation.
    The combinations that do not converge to the transition state are shown in grey. 
    The chosen parameters are marked by a red star.
  }
\end{figure}

In practice, when numerically minimising, the states will jump about slightly which can result in large gradients perpendicular to the optimal movement direction.
To reduce this effect, the size of the gradients in the direction separating the two states are instead calculated:
\begin{equation}
  \abs{\grad E_i} \approx \frac {\abs{(\bm{x}_1 - \bm{x}_2) \cdot \grad E_i}} {\abs{\bm{x}_1 - \bm{x}_2}}.
\end{equation}

The constant parameters $\alpha$ and $\beta$ still must be chosen.
To this end we have tested different parameter choices using the seven-particle cluster and cylindrical buckling examples.
\Cref{fig:paramtest} shows which choices lead to convergence and the speed at which this occurs.
The parameters $\alpha = 10$ and $\beta = 0.1$ are chosen for both converging quickly and being situated far from the regions of non-convergence.
Hence, this choice is likely to still succeed even if the boudaries of the regions were to shift under different systems.
However, the user is able to choose values optimised for their specific system should they wish.


\section{Test potentials and gradients}
To use L-BFGS minisation the gradients of the optimisation function must be known.
For the BITSS potential this is given by \cref{eq:gradBITSS}, but this also requires the gradients of the system under consideration.

\subsection{Particle cluster}
For the Lennard-Jones particle cluster the gradient of the potential is found by considering the gradient of the interaction between each pair of particles,
\begin{equation}
  \frac{\partial E}{\partial \bm{x}_1} = -\frac{\partial E}{\partial \bm{x}_2} = - \frac{24 \epsilon}{r^2}
    \left[ 2 \left(\frac{\sigma}{r}\right)^{12} - \left(\frac{\sigma}{r}\right)^6 \right] (\bm{x}_1 - \bm{x}_2),
\end{equation}
where $\bm{x}_1$ and $\bm{x}_2$ are the positions of the two particles, $r$ is their separation, $\epsilon$ is the interaction strength, and $\sigma$ is the particle radius.

\subsection{Cylindrical buckling}
\begin{figure}[htb]
  \includegraphics{fig/BAHschematic.pdf}
  \caption{\label{fig:BAHschematic}
    Schematic of the bar and hinge model showing the relevent parameters for a single hinge element. $h_i$ and $\bm{\hat{n}}_i$ denote the height and unit normal of each triangle respectively.
  }
\end{figure}
For the cylindrical buckling example we can obtain the gradient by considering each bond and hinge individually.
Using the variables shown in the schematic in \cref{fig:BAHschematic}, the gradient of the stretching energy of the bond between $\bm{x}_2$ and $\bm{x}_3$, with equilibrium length $r^0$ and stretching rigidity $k^S$, is given by,
\begin{equation}
  \frac{\partial E^S}{\partial \bm{x}_2} = - \frac{\partial E^S}{\partial \bm{x}_3} = 
    2 k^S (r - r^0) (\bm{x_2} - \bm{x_3}).
\end{equation}
While the gradients of the bending energy of the hinge, with equilibium angle $\theta^0$ and bending rigidity $k^B$, are,
\begin{align}
  \frac{\partial E^B}{\partial \bm{x}_1} &= k^B \sin(\theta - \theta^0) \frac{\bm{\hat{n}}_a}{h_a}, \\
  \frac{\partial E^B}{\partial \bm{x}_2} &= - k^B \sin(\theta - \theta^0) \left[\frac{\bm{\hat{n}}_a}{h_a} + \frac{\bm{\hat{n}}_a + \bm{\hat{n}}_b}{r}\right], \\
  \frac{\partial E^B}{\partial \bm{x}_3} &= - k^B \sin(\theta - \theta^0) \left[\frac{\bm{\hat{n}}_b}{h_b} + \frac{\bm{\hat{n}}_a + \bm{\hat{n}}_b}{r}\right], \\
  \frac{\partial E^B}{\partial \bm{x}_4} &= k^B \sin(\theta - \theta^0) \frac{\bm{\hat{n}}_b}{h_b}.
\end{align}

\subsection{Striped wetting}
The four sections of the phase-field model detailed in \namecref{eq:gradBITSS}~(6) are obtained using the following equations,
\begin{align}
  E_B[\phi] &= \sum_i \frac{1}{\epsilon} \left( \frac{{\phi_i}^4}{4} - \frac{{\phi_i}^2}{2} + \frac{1}{4} \right) \upDelta V,\\
  E_I[\phi] &= \sum_i \frac{\epsilon}{2} \abs{\grad \phi_i}^2 \upDelta V,\\
  E_S[\phi] &= \sum_j \sqrt{2}\cos\theta_j \left( \frac{{\phi_j}^3}{6} - \frac{\phi_j}{2} - \frac{1}{3} \right) \upDelta S,\\
  E_V[\phi] &= k_V \left[ \sum_i \frac{\phi_i + 1}{2} \upDelta V - V_0 \right]^2,
\end{align}
where the index $i$ includes all of the nodes, while $j$ represents the nodes along the solid surface.
$\upDelta V$ and $\upDelta S$ are, respectively, the volume and solid surface areas contained by each individual node.
$\epsilon$ is the liquid-gas interface width, $\theta_j$ --- the contact-angle with the solid surface, and $V_0$ --- the constrained volume of the liquid drop.
The strength of the volume constraint is parametrised by $k_V$ for which we use a value of $10^4$.
The gradients of these terms are given by,
\begin{align}
  \frac{\partial E_B}{\partial \phi_i} &= \frac{1}{\epsilon} \left( {\phi_i}^3 - \phi_i \right) \upDelta V,\\
  \frac{\partial E_I}{\partial \phi_i} &= \frac{\epsilon}{2} \left[
    \frac{\partial \abs{\grad \phi_i}^2}{\partial \phi_i} +
    \sum_k\frac{\partial \abs{\grad \phi_k}^2}{\partial \phi_i} \right] \upDelta V,\\
  \frac{\partial E_S}{\partial \phi_j} &= \sqrt{2}\cos\theta_j \left( \frac{{\phi_j}^2}{2} - \frac{1}{2} \right) \upDelta S,\\
  \frac{\partial E_V}{\partial \phi_i} &= k_V \left[ \sum_{i'} \frac{\phi_{i'} + 1}{2} \upDelta V - V_0 \right] \upDelta V,
\end{align}
where index $k$ denotes neighboring nodes at which the evaluation of the gradient uses $\phi_i$.

\begin{figure}[htb]
  \includegraphics{fig/DNEBspeedtest.pdf}
  \caption{\label{fig:DNEBspeedtest}
    The rate of convergence to the transition state for (a) a Lennard-Jones seven-particle cluster, (b) cylindrical shell buckling, (c) wetting of a chemically-striped surface.
    Above are the two minimum energy states and the transition state, marked by an asterisk.
    Below is shown the convergence to the transition state as a function of the number of gradient calculations using BITSS (black line) and DNEB.
    The DNEB results are repeated with differing numbers of images along the chain, the number of which is listed in the legend in (c).
  }
\end{figure}


\begin{figure}[htb]
  \includegraphics{fig/clusterdgraphs.pdf}
  \caption{\label{fig:clusterdgraphs}
    Differences in the disconnectivity graphs for the seven-particle cluster when using the discontinuous potential instead of the standard Lennard-Jones potential. The two graphs are horizontally offset for visibility. The energies of the transition states are found to be slightly higher when using the discontinuous potential.
  }
\end{figure}


\end{document}
